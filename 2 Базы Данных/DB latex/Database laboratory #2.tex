\documentclass[12pt]{article}
%For sql code
\usepackage{listings}
%Import geometry for smaller top
\usepackage{geometry}
% Required for inserting images
\usepackage{graphicx}
\graphicspath{{./images/}}
%Header and footer
\usepackage{fancyhdr}
%Language setting
\usepackage[utf8]{inputenc}
\usepackage[T2A]{fontenc}
\usepackage[russian]{babel}

\usepackage{color}
\definecolor{dkgreen}{rgb}{0,0.6,0}
\definecolor{gray}{rgb}{0.5,0.5,0.5}
\definecolor{mauve}{rgb}{0.58,0,0.82}
\lstset{language=SQL,
  basicstyle={\small\ttfamily},
  belowskip=3mm,
  breakatwhitespace=true,
  breaklines=true,
  classoffset=0,
  columns=flexible,
  commentstyle=\color{dkgreen},
  framexleftmargin=0.25em,
  frameshape={}{yy}{}{}, %To remove to vertical lines on left, set `frameshape={}{}{}{}`
  keywordstyle=\color{blue},
  numbers=none, %If you want line numbers, set `numbers=left`
  numberstyle=\tiny\color{gray},
  showstringspaces=false,
  stringstyle=\color{mauve},
  tabsize=3,
  xleftmargin =1em
}

\geometry{a4paper,
 total={170mm,257mm},
 left=20mm,
 top=30mm,
 bottom=25mm,
 }

\fancypagestyle{first style}
{
\chead{\footnotesize{Санкт-Петербургский Национальный Исследовательский Университет ИТМО\\Факультет Программной Инженерии и Компьютерной Техники}}
\cfoot{\footnotesize{Санкт-Петербург 2023г.}}
\renewcommand{\headrulewidth}{0pt}
}

\begin{document}

\pagestyle{fancy}
$\thispagestyle{first style}$

\centering{\includegraphics[scale=0.5]{LogoITMO}}

\vspace{25mm}

\centering{Вариант №1537\\Лабораторная работа№2\\По дисциплине\\Базы Данных}

\vspace{50mm}

\begin{flushright}
Выполнил студент группы P3115:\\Хромов Даниил Тимофеевич\\
\vspace{5mm}
Преподаватель:\\Горбунов Михаил Витальевич\\Николаев Владимир Вячеславович
\end{flushright}

\newpage

\pagestyle{empty}
\flushleft{\large{\textbf{1. Текст Задания}}}\\
По варианту, выданному преподавателем, составить и выполнить запросы к базе данных "Учебный процесс".\\\vspace{5mm}
Составить запросы на языке SQL (пункты 1-7).

Сделать запрос для получения атрибутов из указанных таблиц, применив фильтры по указанным условиям:\\
1. Н\_ТИПЫ\_ВЕДОМОСТЕЙ, Н\_ВЕДОМОСТИ.\\
Вывести атрибуты: Н\_ТИПЫ\_ВЕДОМОСТЕЙ.ИД, Н\_ВЕДОМОСТИ.ДАТА.\\
Фильтры (AND):\\
a) Н\_ТИПЫ\_ВЕДОМОСТЕЙ.НАИМЕНОВАНИЕ < Ведомость.\\
b) Н\_ВЕДОМОСТИ.ЧЛВК\_ИД = 163249.\\
Вид соединения: RIGHT JOIN.\\
2. Сделать запрос для получения атрибутов из указанных таблиц, применив фильтры по указанным условиям:\\
Таблицы: Н\_ЛЮДИ, Н\_ОБУЧЕНИЯ, Н\_УЧЕНИКИ.\\
Вывести атрибуты: Н\_ЛЮДИ.ИМЯ, Н\_ОБУЧЕНИЯ.НЗК, Н\_УЧЕНИКИ.ГРУППА.\\
Фильтры: (AND)\\
a) Н\_ЛЮДИ.ФАМИЛИЯ < Ёлкин.\\
b) Н\_ОБУЧЕНИЯ.НЗК = 001000.\\
c) Н\_УЧЕНИКИ.ИД = 150308.\\
Вид соединения: RIGHT JOIN.\\
3.Вывести число фамилий и отчеств без учета повторений.\\
При составлении запроса нельзя использовать DISTINCT.\\
4. Найти группы, в которых в 2011 году было ровно 5 обучающихся студентов на ФКТИУ.\\
Для реализации использовать соединение таблиц.\\
5. Выведите таблицу со средним возрастом студентов во всех группах (Группа, Средний возраст), где средний возраст равен максимальному возрасту в группе 1100.\\
6. Получить список студентов, зачисленных ровно первого сентября 2012 года на первый курс очной или заочной формы обучения. В результат включить:\\
номер группы;\\
номер, фамилию, имя и отчество студента;\\
номер и состояние пункта приказа;\\
Для реализации использовать соединение таблиц.\\
7. Вывести список студентов, имеющих одинаковые имена, но не совпадающие даты рождения.\\
\vspace{5mm}
\flushleft{\large{\textbf{2. Реализация запросов на SQL}}}\\
\lstinputlisting[language=SQL]{code/lab2.sql}

\flushleft{\large{\textbf{3. Вывод}}}\\
При выполнении лабораторной работы я познакомился с основными функциями языка SQL и диалекта PostgreSQL. Научился писать запросы, получать, агрегировать, отсеивать и сортировать полученные данные с использованием различны синтаксических конструкций языка. В результате был освоен язык DML SQL, предназначенный для работы с данными, хранящимися внутри базы данных.

\end{document}